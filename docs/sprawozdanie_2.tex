\documentclass[12pt]{article}
\usepackage{polski}
\usepackage[utf8]{inputenc}
\usepackage{graphicx}
\usepackage[normalem]{ulem}
\usepackage{listings}
\usepackage{hyperref}

\title{%
    Snake \\
    \large Programowanie II - projekt zaliczeniowy \\
    Wydział Matematyki Stosowanej \\
    Politechnika Śląska \\}

\author{Piotr Skowroński, Bartłomiej Pacia}
\date{Czerwiec 2022}

\begin{document}

\maketitle

\section{Wstęp}

“Snake” jest klonem popularnej “gry w węża”. Rozgrywka toczy się na kwadratowej
planszy. Na początku gracz startuje z wężem o długości 1 (tzn. zajmującym 1
kratkę). Jednocześnie w losowej kratce na mapie pojawia się punkt. Zadaniem
gracza jest takie kierowanie swoim wężem przy użyciu klawiszy strzałek na
klawiaturze, by zebrać punkt. Zebrany punkt znika, a długość węża gracza
zwiększa się o 1. Następnie pojawia się nowy punkt i cały proces zaczyna się od
nowa. Gracz przegra, jeśli uderzy głową węża w jego własne ciało lub wyjdzie
poza mapę. Uniknięcie tej pierwszej sytuacji staje się coraz trudniejsze wraz ze
wzrostem długości węża.

\section{Wymagania}

\textbf{Pogrubione} elementy listy oznaczają nowe wymagania, niezdefiniowane w
pierwotnych Założeniach do Projektu, które zostały zrealizowane.
\begin{itemize}
    \item ekran menu z 3 przyciskami (“Graj”, "Info" i “Wyjdź”)

    \item ekran gry z 2 przyciskami (“Pauza”, i “Wyjdź”), licznikiem
          obecnej liczby punktów oraz najwyższej liczby punktów

    \item zbieranie punktów i w konsekwencji wydłużanie się węża

    \item zapisywanie najwyższej liczby punktów (high score) do pliku i
          możliwość pobicia tego rekordu

    \item \textbf{możliwość podania rozmiaru planszy i interwału czasowego
              między ruchami jako argumenty do programu}

\end{itemize}


\section{Przebieg realizacji}

Staraliśmy pisać kod z użyciem nowych funkcjonalności zapewnianych przez
standardy C++11, C++14 i C++17. W szczególności \textit{smart pointers}
zamiast \textit{raw pointers} oraz przekazywania argumentów do funkcji i metod
przez referencję. Użyliśmy też kontenera \textit{std::vector} oraz funkcji z
biblioteki \textit{algorithm}.

\subsection{Użyte biblioteki}

Użyliśmy biblioteki standardowej C++ oraz biblioteki SFML, która udostępnia
podstawowe funkcje do obsługi grafiki.

\subsection{Użyte narzędzia}

Użyliśmy systemu kontroli wersji \textit{git} oraz serwisu \textit{GitHub} do
hostowania repozytorium projektu.

Użyliśmy też narzędzia \textit{clang-format}, aby utrzymywać spójny styl kodu.
Użyty przeze nas styl to \textit{Chromium}.

\section{Instrukcja użytkownika}

\subsection{Budowanie projektu}
Aby skompilować projekt, potrzebny jest:
\begin{itemize}
    \item kompilator C++, np. \textit{g++} lub \textit{clang++}

    \item program \textit{make}

    \item źródła biblioteki SFML, którą można zainstalować stworzonym przez
          nas skryptem \textit{install\_sfml} dostępnym w projekcie
\end{itemize}

Po spełnieniu powyższych wymagań możemy w głównym katalogu projektu uruchomić
program \textit{make}


\begin{lstlisting}
$ make snake
\end{lstlisting}

\subsection{Uruchamianie gry}

Grę należy uruchamiać z linii komend. Pozwala to na wyświetlanie pomocy oraz
przekazanie mu argumentów wpływających na wygląd i gameplay.

Aby uruchomić grę z domyślnymi ustawieniami (ruch węża co 500 ms i plansza o
rozmiarze 16 x 16 kratek), nie trzeba podawać żadnych argumentów.


\begin{lstlisting}
$ ./snake
\end{lstlisting}

Aby uruchomić grę z wężem poruszającym się co 100 ms i planszą o rozmiarze 128 x
128 kratek, należy podać następujące argumenty:

\begin{lstlisting}
$ ./snake --interval 100 --grid-size 128
\end{lstlisting}

Aby poruszać wężem należy używać strzałek. \\~\\
Repozytorium projektu na GitHubie:
\href{https://github.com/bartekpacia/snake}{github.com/bartekpacia/snake}

\end{document}
